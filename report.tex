% 'draft' mode can be used to speed up compilation
\documentclass[twoside,final]{Phenikaa-report}
\usepackage{codespace}

% Draft watermark
% https://github.com/callegar/LaTeX-draftwatermark

% Encodings
\usepackage{gensymb,textcomp}

% Better tables
% Wide tables go to https://tex.stackexchange.com/q/332902
\usepackage{array,longtable,multicol,multirow,siunitx,tabularx}

% Better enum
\usepackage{enumitem}

% Graphics
\usepackage{caption,float}

% Add options for figures, like max width, framing, etc.
\usepackage[export]{adjustbox}

% References
% Use \Cref{} instead of \ref{}
\usepackage[nameinlink]{cleveref}

% FOR DEMONSTRATION PURPOSES, REMOVE IN PRODUCTION
\usepackage{mwe}

% Sub-preambles
% https://github.com/MartinScharrer/standalone

% Configurations
\coursename{Laboratorio de Electrónica Analógica}
\reporttype{Informe}
\title{Control analogico satelital}
\advisor{& Juan Pablo Adriach&}
\stuname{%
  & Simón Aulet &  \\
  & Juan Nicolás Roccasalvo & \\
  & Nuria Belén Paredes &  \\
}

% Allow page breaks inside align* environment
%\allowdisplaybreaks{}

% Makes a lot of things blue, avoid at all costs
%\everymath{\color{blue}}

% Set depth of numbering for counters
\AtBeginDocument{\counterwithin{lstlisting}{section}}

% Rename some sections
%\AtBeginDocument{\renewcommand*{\contentsname}{Contents}}
%\AtBeginDocument{\renewcommand*{\refname}{References}}
%\AtBeginDocument{\renewcommand*{\bibname}{References}}

% Custom commands
%\newcommand*\mean[1]{\bar{#1}}

% Allow page breaks inside align* environment
%\allowdisplaybreaks{}

% Makes a lot of things blue, avoid at all costs
%\everymath{\color{blue}}

% Set depth of numbering for counters
\AtBeginDocument{\counterwithin{lstlisting}{section}}

% Rename some sections
%\AtBeginDocument{\renewcommand*{\contentsname}{Contents}}
%\AtBeginDocument{\renewcommand*{\refname}{References}}
%\AtBeginDocument{\renewcommand*{\bibname}{References}}

% Custom commands
%\newcommand*\mean[1]{\bar{#1}}
\AtBeginDocument{\renewcommand{\contentsname}{Tabla de contenidos}}
\AtBeginDocument{\renewcommand{\listfigurename}{Indice de figuras}}
\AtBeginDocument{\renewcommand{\listtablename}{Indice de Tablas }}
\AtBeginDocument{\renewcommand{\tablename}{Bảng}}
\AtBeginDocument{\renewcommand{\figurename}{Hình}}
\AtBeginDocument{\renewcommand{\refname}{Các nguồn tài liệu than khảo}}

\begin{document}
\coverpage%

%\section*{Member list \& Workload}
%\newcounter{memberrowno}
%\setcounter{memberrowno}{0}
%\begin{center}
%  \begin{tabular}{>{\stepcounter{memberrowno}\thememberrowno}llcc}
%    \toprule
%    \multicolumn{1}{c}{\textbf{No.}} & \textbf{Full name} & \textbf{Student ID} & \textbf{Contribution} \\
%    \midrule
%                                     & h                  & xxxxxxx             & 100\%                       \\
%                                     & h                  & xxxxxxx             & 100\%                       \\
%    \bottomrule
%  \end{tabular}
%\end{center}
%\clearpage

\tableofcontents
\listoffigures
\listoftables
\lstlistoflistings{}

\clearpage
\section{Introducción}


\begin{center}
  \mintinline{latex}{\include{filename}} = \mintinline{latex}{\clearpage \input{filename} \clearpage}
\end{center}

\section{Metodología}
There are no definite rules for label prefixes, but you can use the following as a guideline.
\begin{itemize}
  \item \textbf{chap:} for chapters
  \item \textbf{sec:} for sections
  \item \textbf{subsec:} for subsections
  \item \textbf{eq:} for equations
  \item \textbf{fig:} for figures
  \item \textbf{tab:} for tables
  \item \textbf{enum:} for enumerators and items
  \item \textbf{fn:} for footnotes
  \item \textbf{lst:} for listings
  \item \textbf{alg:} for algorithms
  \item \textbf{app:} for appendices
\end{itemize}

The \mintinline{latex}{\caption} macro increases the used counter and sets the current label text which is used by \mintinline{latex}{\label}.
If you use \mintinline{latex}{\label} before it the old label text is used instead, which leads to a wrong number.
Always use \mintinline{latex}{\label} after \mintinline{latex}{\caption} and not before or in it.

That said, conventions are just conventions, and you can use whatever you want as long as you are consistent.

\clearpage

\section{Introducción }
Diseñamos la computadora de a bordo de un sistema de control satelital completamente analógico. El desarrollo del proyecto se desglosó en 14 laboratorios, cada uno de los cuales aportó componentes relevantes para el aprendizaje del diseño analógico. Esto incluyó la implementación de funciones específicas para la placa, la implementación de conceptos de teoría de control, la integración de sistemas electrónicos, previa simulación y, finalmente, la puesta a punto.

\begin{figure}[H]
  \includegraphics[max width=0.99\linewidth]{graphics/0.3Dboard1.png}
  \caption{PCB Terminado}%
  \label{fig:pcb 3D}
\end{figure}

\section{Control térmico}
\section*{Objetivo}
\section*{Implementación}

\subsection*{Simulaciónes}

\subsubsection*{Planteo matemático}
Para las simulaciónes se parte de la ecuación de schmitt trigger:

\begin{align*}
V_{in}L  &= \frac{R_1}{R_1 + R_2} \left(V_{OL} - V_{ref}\right) + V_{ref} \\
V_{inH}  &= \frac{R_1}{R_1 + R_2}\left(V_{OH} - V_{ref} + V_{ref}\right) \\
H        &= \frac{R_1}{R_1 + R_2} \left(V_{OH} - V_{OL}\right)
\end{align*}

Para elegir los valores, se simula el sistema de ecuaciónes en Python

\begin{lstlisting}[
    language=python,caption={Simulacion histeresis},name=simulacion histeresis,label=lst:simulacion histeresis
  ]

    from sympy import symbols, Eq, solve
    import matplotlib.pyplot as plt

    #Generacion de simbolos
    VinL, VinH, H, R1, R2, VOL, VOH, Vref = symbols('VinL VinH H R1 R2 VOL VOH Vref')

    #Implementacion de ecuaciones
    eq_min = Eq(VinL, R1/(R1+R2) * (VOL - Vref) + Vref)
    eq_max = Eq(VinH, R1/(R1+R2) * (VOH - Vref) + Vref)
    eq_H   = Eq(H,    R1/(R1+R2) * (VOH - VOL))

    #Se sobreescriben los valores conocidos dejandose 3 incognitas
    VinL_val = VinL
    VinH_val = VinH
    H_val    = H
    R1_val   = 5e3
    R2_val   = 100e3
    VOL_val  = 0
    VOH_val  = 9
    Vref_val = 5

    #Se resuelve el circuito para v superior, inferior e histeresis
    sol = solve(
        (eq_min.subs({R1: R1_val, R2:R2_val, VOL:VOL_val, Vref:Vref_val}),
         eq_max.subs({R1: R1_val, R2:R2_val, VOH:VOH_val, Vref:Vref_val}),
         eq_H.subs({  R1: R1_val, R2:R2_val, VOH:VOH_val, VOL:VOL_val})),
        (VinL, VinH, H)
    )
    \end{lstlisting}


\begin{minipage}{0.5\textwidth}
\begin{figure}[H]
  \includegraphics[max width=0.9\linewidth]{graphics/1.6.hysteresis_calculations.png}
  \caption{Ploteo de límites para histéresis}%
  \label{fig:plot-histeresis}
\end{figure}
\end{minipage}
\begin{minipage}{0.4\textwidth}
    El código nos determina límites de voltaje para encendido y apagado ($V_{in}$) de la salida del amplificador, así como los valores superiores e inferior de voltaje que son $VCC$ y $0$ respectivamente. El ancho resulta en $430mV$
\end{minipage}

\subsubsection*{Simulación en LTSPice}
A partir de esto, se realizan las simulacióne en LTSpice. No del circuito final, si no de los tres casos posibles de funcionamiento del schmitt trigger:
\begin{itemize}
    \item Histeresis con inversión
    \item Histéresis sin inversión
    \item Comparador sin histéresis
\end{itemize}

El circuito implementado en LTSpice es el siguiente:

\begin{figure}[H]
  \includegraphics[max width=0.9\linewidth]{graphics/1.1.circuit.png}
  \caption{Circuito implementado en LTSpice}%
  \label{fig:Implementación en LTSpice de Schmitt Trigger}
\end{figure}
Se puede apreciar cómo se hacen 3 circuitos para cada uno de los casos detallados más arriba
\subsubsection*{Resultados simulación}

A continuación, se muestran los resultados de cada simulación realizada. Las primeras tres imágenes muestran la forma de onda de los 3 casos medidos.

\begin{figure}[H]
  \includegraphics[max width=0.9\linewidth]{graphics/1.2.inverting_schmitt.png}
  \caption{Salida de inverting schmitt trigger}%
  \label{fig:Schmitt Trigger inversor}
\end{figure}

\begin{figure}[H]
  \includegraphics[max width=0.9\linewidth]{graphics/1.3.non-inverting_schmitt.png}
  \caption{Salida de non-inverting schmitt trigger}%
  \label{fig:Schmitt Trigger no-inversor}
\end{figure}

\begin{figure}[H]
  \includegraphics[max width=0.9\linewidth]{graphics/1.4.non-hysteretic_comparator.png}
  \caption{Salida sin histéresis}%
  \label{fig:Salida no Schmitt Trigger}
\end{figure}

Finalmente, se incluye una prueba de la amplitud de la histéresis para verificar que la simulación concuerda con los calculos teóricos

\begin{figure}[H]
  \includegraphics[max width=0.9\linewidth]{graphics/1.5.hysteresis_characteristics.png}
  \caption{Medición características de la histéresis}%
  \label{fig:caracteristicas histeresis}
\end{figure}
Se usan marcadores para determinar el ancho de la histéresis $(H)$. Ese valor se resalta en azúl en la ventana de marcadores. Se puede apreciar que da $425mV$, totalmente en línea con lo simulado en Python

\subsection*{Análisis y consideraciones de diseño}
El ancho de histéresis obtenido fue aproximadamente $H \approx 430\,mV$ en la entrada del comparador.
Dado que el sensor LM35 posee una sensibilidad de $10\,mV/^\circ C$ y la etapa amplificadora tiene una ganancia de 5, esto implica que el ancho real de histéresis en temperatura es:

\[
\Delta T = \frac{H}{5 \cdot 10\,mV/^\circ C} \approx 8.6^\circ C
\]

Este valor resulta adecuado para evitar conmutaciones erráticas debidas a pequeñas fluctuaciones o ruido cercano al umbral.

Por otro lado, en la implementación práctica deben considerarse tolerancias de resistencias y no idealidades del LM358, como tensión de saturación no perfectamente rail-to-rail y pequeño offset de entrada.
Estas variaciones pueden generar leves diferencias respecto a los valores teóricos, aunque no afectan demasiado el funcionamiento general del sistema.

En conjunto, el diseño presenta un comportamiento coherente con el análisis teórico, la simulación y la práctica experimental.

\subsection*{Implementación en PCB}
\begin{figure}[H]
  \includegraphics[max width=0.99\linewidth]{graphics/1.schematic.pdf}
  \caption{Esquemático}%
  \label{fig:Esquematico control térmico}
\end{figure}

Para la implementación en el PCB se diseñan 2 bloques diferentes correspondientes a dos laboratorios.
Ambos bloques encienden una señal de control en el board principal y envían una señal de control a la salida.
High temp alert system envía señal de control cuando la temperatura está por encima del valor establecido y low temp heater control hace lo inverso.

\vspace{6pt}
\begin{minipage}{0.3\textwidth}
\flushleft{
El stetpoint de temperatura se establece mediante potenciómetros analógicos ubicados en el main board.
En la figura se muestra el sector del PCB donde se ubican los potenciómetros para setear high y low temp respectivamente más los LED indicadores.}
\end{minipage}%
\begin{minipage}{0.5\textwidth}
\begin{figure}[H]
  \includegraphics[max width=0.99\linewidth]{graphics/1.0.pcb_control.png}
  \caption{Controles manuales en PCB}%
  \label{fig:control manual en PCB}
\end{figure}
\end{minipage}

\subsection*{Evaluación de la etapa de control térmico}

La etapa de control térmico permitió validar el funcionamiento del sistema desde el sensor hasta la señal de salida.
El análisis teórico de los umbrales y del ancho de histéresis mostró coherencia con las simulaciones realizadas, y el armado del modelo confirmó un comportamiento estable del comparador ante variaciones próximas al punto de disparo.

La incorporación de histéresis resultó fundamental para evitar oscilaciones no deseadas, garantizando una conmutación limpia y predecible en condiciones cercanas al umbral.

\section{Orientación solar}
\subsection{Objetivo}

Determinar la orientación del satélite y el evento de un eclipse mediante la luz recibida en sensores luminicos, activando salidas de control correspondientes

\subsection{Implementación}

\subsubsection{Planteo}
Para determinar la orientación del sistema respecto a la luz solar usamos un detector basado en sensores LDR.

Dispusimos cuatro LDR formando los vértices de un cuadrado, aunque debido a su comportamiento idéntico simulamos solo uno de los canales.

El principio de funcionamiento se basa en utilizar la variación resistiva del LDR en función de la iluminación incidente. En la simulación esta variación fue modelada asignando al LDR una resistencia dependiente de tensión:

\[
R_{LDR} = V(VR) \cdot 10000
\]

donde la fuente auxiliar $V_3$ permite simular un cambio suave en la iluminación.

\subsubsection{Consideraciones de funcionamiento}

El circuito usa un comparador para determinar si el nivel de iluminación supera un umbral ajustable mediante un preset modelado por las resistencias $RPOT1$ y $RPOT2$.

El uso de cuatro sensores LDR ubicados en los puntos cardinales permite obtener una estimación de la orientación relativa del satélite respecto a la fuente luminosa.
Una mayor iluminación sobre un sensor implica una menor resistencia del LDR correspondiente, generando una variación en la tensión de entrada del comparador y activando la salida asociada.

La detección de eclipse se define cuando los cuatro sensores presentan niveles
de iluminación por debajo del umbral configurado. En esta condición, las cuatro salidas individuales permanecen en el mismo estado lógico, lo que permite activar una señal global de eclipse.
Debe considerarse que los LDR presentan una respuesta no lineal y dependiente
de la intensidad luminosa, así como variaciones entre dispositivos.
Por este motivo, se incorporan potenciómetros de ajuste que permiten calibrar
los umbrales de disparo y compensar diferencias entre sensores.

De este modo, el sistema permite una detección de condiciones de
iluminación y orientación básica mediante un esquema analógico simple.

\subsubsection{Simulación}
Siendo un circuito muy simple, se realiza unicamente una simulación en la cual se hace variar una resistencia (simulando el LDR) cuyo valor depende de una fuente de voltaje $R = V(VR)\cdot 1M\Omega$. El circuito se muestra a continuación

El análisis temporal muestra la variación progresiva del voltaje en el LDR y la correspondiente conmutación en la salida del comparador.

Debido a que la variación lumínica ocurre de manera lenta, no implementamos histéresis en el comparador, ya que no se observaron oscilaciones en la zona de umbral.

\begin{figure}[H]
  \includegraphics[max width=0.9\linewidth]{graphics/2.1.circuit.png}
  \caption{Simulación del LDR}%
  \label{fig:simulacion LDR}
\end{figure}

La salida es un simple comparador:
\begin{figure}[H]
  \includegraphics[max width=0.9\linewidth]{graphics/2.3.waveform(annotated).png}
  \caption{Salida del LDR}%
  \label{fig:salida LDR}
\end{figure}

\subsubsection{Implementación en el PCB}

Para el esquemático, se usa el mismo circuito comparador 4 veces. Notese que se tienen 4 LEDs, 4 Setpoints y una salida de Eclipse. La detección de eclipse se ajusta mediante un setpoint que se setea al punto en el cual los 4 sensores están apagados (las cuatro salidas en alto)
\begin{figure}[H]
  \includegraphics[max width=0.9\linewidth]{graphics/2.schematic.pdf}
  \caption{Esquematico de detección de orientacion}%
  \label{fig:esquematico orientacion}
\end{figure}


En cuanto al PCB, se agruparon los setpoints para los LDR en un sector junto con el LED indicador de detección de eclipse.
Los 4 sensores se implementan en 4 PCBs separados con conectores para las señales hacia el main board

\begin{minipage}{0.49\textwidth}
\begin{figure}[H]
  \includegraphics[max width=0.9\linewidth]{graphics/2.4.pcb.png}
  \caption{Selectores e indicador de eclipse en el PCB}%
  \label{fig:control eclipse}
\end{figure}
\end{minipage}
\begin{minipage}{0.49\textwidth}
\begin{figure}[H]
  \includegraphics[max width=0.9\linewidth]{graphics/2.5.pcb_sensores.png}
  \caption{Sensores exteriores de luz}%
  \label{fig:sensores exteriores}
\end{figure}
\end{minipage}

\section{Despliegue de paneles}
\subsection*{Objetivo}
Diseñar un sistema de control para el despliegue automático de paneles solares o antenas.
Relación con Sistemas Reales: Este circuito representa el mecanismo de despliegue de paneles solares o antenas en satélites, que deben activarse de forma precisa y detenerse automáticamente al alcanzar su posición final.
\subsection*{Implementación}

El despliegue es realizado por un servomotor, donde su debido control nos garantiza la presicion en la activacion, y recorrido final (control PWM).
El sistema de despliegue se conforma de tres etapas, donde la primera tal como lo dice la grafica es de oscilación, la segunda de activación y la última es la física.
\begin{figure}[ht]
    \centering
    \begin{tikzpicture}[auto, node distance=1.5cm, >=latex']
        % 1. Definición de los Bloques (Nodos)
        \node [input, name=input] {};
        \node [block, right=of input] (oscilador) {Oscilador Wien};
        \node [block, right=of oscilador] (comparador) {Comparador};
        \node [block, right=of comparador] (servomotor) {Servomotor};
        \node [output, right=of servomotor] (output) {};

        % 2. Conexión de los bloques con flechas
        \draw [->] (input) -- node {V DC} (oscilador);
        \draw [->] (oscilador) -- node {Seno} (comparador);
        \draw [->] (comparador) -- node {PWM} (servomotor);
        \draw [->] (servomotor) -- node {Posición} (output);
    \end{tikzpicture}
    \caption{Diagrama en bloques del despliegue.}
\end{figure}
\\La idea en esta implementacion es diseñar un controlador analógico para un servomotor, la única manera de lograrlo es mediante el conocido PWM, en otras palabras es hacerle llegar una determidada cantidad de pulsos por segundo para su activacion y modularlos para su control en grados.
Para ello decidimos transformar una señal sinudoidal en una de pulso cuadrado mediante un amplificador operacional en una configuración de comparador.
\subsubsection*{Oscilador}
Ecuación que rige el funcionamiento de un oscilador:
\\$A_{f} = \frac{V_o}{V_i} = \frac{A}{1 - A\beta}$
\\A diferencia de un feedback convencional con entrada de referencia y salida, el oscilador se caracteriza por no poseer entrada es decir, tenemos un bloque de ganancia A con una salida $V_{0}$ realimentada mediante un bloque de ganancia de $\beta$ hacia el bloque A directamentel, sin pasar por ningun sumador entre medio. 
\\Esta caracterizacion hace que la ecuacion de ganancia de feedback necesite a $A\beta =1$ para hacer la ganancia infinita y entrar en el bucle de osclación. Entonces para que la placa cumpla con este requisito de un sencillo despeje tenemos que que la ganancia de feedback expresada con Laplace queda en: 
\\$A(s) = \frac{V_o(s)}{V_f(s)} = 1 + \frac{R_F}{R_1}$

R1 y Rf están asociadas a la parte de realimentación negativa del amplificador. \\
 $(1 + \frac{R_F}{R_1}) \frac{RCs}{R^2 C^2 s^2 + 3RCs + 1} = 1$\\
si sustituimos jw por s en la ecuación obtenemos que:\\
 $(1 + \frac{R_F}{R_1}) \frac{RC(jw))}{R^2 C^2 (jw))^2 + 3RC(jw) + 1} = 1$\\
  
  Luego de despejar de la parte imaginaria w, quedamos en los siguientes valores:
 $1+\frac{Rf}{R1}=3$\\
 donde necesariamente nos queda que $\frac{Rf}{R1}=2$\\
 Lo cual para la práctica impone ésta condición de frontera para garantizar la estabilidad del oscilador, es decir que no diverga ni corverga la onda en ningun periodo de operacion. Por recomendación del profesor la resistencia de Rf fue implementada por preset para poder dar ese ajuste fino que en calculos no se puede manejar a causa de las perturbaciones externas.
 
\begin{figure}[H]
  \includegraphics[max width=0.8\linewidth]{graphics/wien.jpg}
  \caption{Circuito oscilador sinusoidal}%
  \label{fig:plot-histeresis}
\end{figure}

\subsubsection*{Comparador}
\begin{figure}[H]
  \includegraphics[max width=0.8\linewidth]{graphics/comp.png}
  \caption{Circuito oscilador sinusoidal}%
  \label{fig:circuito oscilador}
\end{figure}
El comparador consta de una señal de referencia en la pata inversora que compara constantemente con la tension de la señal sinusoidal inyectada en la pata no inversora, entonces cuando la señal alcanza y supera el voltaje de comparación, la salida permanece en estado alto en toda la fracción de la cresta que hasta que este voltaje es menor a la referencia, entonces obtenemos el pulso cuadrado, que ser repite cada 50Hz, ya que la comparación aparece en éste periodo de 20mS. Modulando el pulso con un preset que ajuste la referencia estamos sobrados para hacer todo el barrido del servomotor ya que el rango de valores de 0 a 180° equivale a 1 a 2mS en la señal.



\subsection*{Pruebas de laboratorio}
\begin{figure}[H]
  \includegraphics[max width=0.9\linewidth]{graphics/osc.png}
  \caption{Prueba en osciloscopio}%
  \label{fig:prueba osciloscopio}
\end{figure}
Seteamos los valores de resistencias y capacitivos en la parte de feedback a un mismo R y un miscmo C para poder despejar una única frecuencia de oscilacion en este caso la frecuencia de operación es 50Hz para mover el servomotor.\\

\subsection*{Análisis del control PWM}

El servomotor utilizado requiere una señal PWM con frecuencia cercana a 50\,Hz 
(período aproximado de 20\,ms). Dentro de cada período, el ancho de pulso determina la posición angular del eje, siendo típicamente 1\,ms el extremo mínimo y 2\,ms el máximo (0° a 180°).

La señal sinusoidal generada por el oscilador Wien es convertida en una señal cuadrada mediante un comparador. La tensión de referencia aplicada al comparador determina el punto de cruce con la señal senoidal y, por lo tanto, el ancho del pulso generado.
Al variar dicha referencia mediante un preset, se modifica el tiempo durante el cual la señal permanece en nivel alto dentro de cada período de 20\,ms, permitiendo controlar la posición del servomotor de manera analógica.

Este enfoque permite implementar un control PWM simple sin necesidad de 
microcontroladores, utilizando solo bloques analógicos.

\subsection*{Problemas y conclusiones}
En el oscilador Puente Wien la ganancia RF/R1 =2 tenia que ser un valor exacto al principio con un valor teórico exacto no pudimos llegar a la oscilacion como tal que primero nos aparecia una linea continua en el osciloscopio (convergencia rápida) o directamente no aparecía nada (divergencia rápida). La recomendación del profesor de colocar el preset fue de gran utilidad para visualizar el punto justo de oscilación en pantalla a medida que ibamos girando la perilla.




\section{Verificación despliegue}
\subsection*{Objetivo}

Hacer un sistema de verificación analógica de despliegue exitoso de paneles solares.

\subsection*{Implementación}

Para verificar el correcto despliegue, se comparan dos señales: La señal de selección
A\_PANEL\_OPEN / CLOSE\_LATCH\_I con la señal de sensor A\_PANEL\_OPEN / CLOSE\_SENSOR\_I.
Al haber un cambio en la señal de control de selección se dispara un timer. Si la señal de selección (latcheada) tiene una diferencia con la señal de sensor durante un tiempo mayor al establecido con el timer 555, se asume un error y se envía señal de alarma. Si las señales se igualan antes que el timer llegue a su límite, se asume correcto funcoinamiento.
A cotinuación se muestra el esquemático con cada bloque:
\begin{figure}[H]
  \includegraphics[max width=0.99\linewidth]{graphics/4.schematic.pdf}
  \caption{Esquematico completo}%
  \label{fig:esquematico verificación}
\end{figure}

\begin{itemize}
  \item \textbf{Signal-to-pulse trigger:} La función de este bloque es convertir la señal continua de selección en un pulso invertido para activar el 555 ya que éste necesita pulsos para activarse. Para lograrlo, se usa un capacitor el cual, al llegar una señal contínua al comparador, genera una diferencia temporal manteniendose la entrada negativa a mayor voltaje que la positiva. Esto hace que el opamp tenga una salida baja. Luego, una vez el capacitor se estabiliza la salida vuelve a estado alto. En las simulaciónes se puede ver este funcionamiento. El sistema se duplica para poder activar el timer con ambas entradas; apertura y cierre de paneles.
  \item\textbf{555 timing:} Timer 555 configurado en modo monoestable con una duración de 10 segundos approx.
  \item\textbf{Reset signal mux:} La señal de reset es tomada de los sensores. Cuando un sensor se activa (el panel está en posición) se activa el reset. Cuando se selecciona apertura paneles, el mux deja pasar la señal del sensor de apertura. De esta manera, cuando el panel está totalmente abierto, el sensor de apertura se activa y a su vez activa el reset. Como el reset es ativo bajo, esta señal se invierte en el mismo mux con el transistor $Q601$. Idem para cierre.
  \item\textbf{Alarm set logic:} Activa la alarma cuando la salida $Q$ del timer está en bajo (activa) y el reset no está activado. Esto indica que se superó el tiempo de espera para apertura. Si el sensor indica que el panel llegó a su posición, el reset se activa (bajo) y el comparador se mantiene en cero. Si el timer se agota antes que el sensor llegue a la posición, el comparador da salida en alto activando la alarma.
  \item\textbf{Panel status logic:} Indicador que informa el estado de los paneles. Cuando el latch seleccióna apertura y el sensor indica que está abierto, sale la señal de panel abierto. Lo mismo para cerrado
\end{itemize}

\subsubsection*{Simulaciónes}
Se arma el circuito en LTSpice para una  sola señal como se muestra a continuación.

\begin{figure}[H]
  \includegraphics[max width=0.99\linewidth]{graphics/4.1.circuit.png}
  \caption{Circuito completo en LTSpice}%
  \label{fig:Simulacion verificacion}
\end{figure}

En la primer simulación se puede ver el latch-to-pulse funcionando. Inicialmente, la señal continua (verde) esta en estado bajo y pasa a alto, generandose un pulso (invertido) en la salida (azul). Luego a los 6 segundos la señal continua pasa a estado bajo sin que se altere la salida. A los 8 segundos la continua pasa a estado alto, generandose otro pulso negativo en la salida.

\begin{figure}[H]
  \includegraphics[max width=0.99\linewidth]{graphics/4.3.latch-to-pulse(annotated).png}
  \caption{Forma de onda de entrada y salida del signal-to-pulse}%
  \label{fig:signal-to-pulse simulation}
\end{figure}

Luego pasamos a la salida del 555. La waveform roja es la salida (activo bajo). Al inicio se dispara el 555 con un pulso de trigger (azul) y a los 6 segundos se activa el reset. Esto genera una salida activa pero no se dispara la alarma por el bloque alarm-logic explicado más arriba.\\
Luego, a los 5 segundos, se vuelve a activar el timmer simulando otro cambio. Esta la señal de reset no llega, frente a lo cual, en los 19 segundos, se activa la salida. La lógica de la alarma hará que, al activarse la salida del 555 y no el reset del sensor, se dispare la alarma.

\begin{figure}[H]
  \includegraphics[max width=0.99\linewidth]{graphics/4.5.out(annotated).png}
  \caption{Forma de onda de entrada, reset y salida del 555}%
  \label{fig:555 simulation}
\end{figure}

Finalmente, está la lógica de la alarma, en la cual se hacen dos pruebas:
\begin{itemize}
    \item A los 6 segundos, la salida y reset están bajos (activos) frente a lo cual la alarma no se activa
    \item A los 19 segundos, Trigger está bajo y reset alto, lo cual activa la alarma.
\end{itemize}
Esta waveform está sincronizada con la anterior (es la misma simulación), con lo cual se puede ver el funcionamiento completo.

\begin{figure}[H]
  \includegraphics[max width=0.99\linewidth]{graphics/4.5.out(annotated).png}
  \caption{Forma de onda de lógica de alarma}%
  \label{fig:alarm-logic simulation}
\end{figure}

\subsection*{Principio de funcionamiento}
Lo que hace este sistema es implementar una verificación temporal de coherencia entre la orden de despliegue y la respuesta física del mecanismo.

Cuando se emite una señal de apertura o cierre, el bloque signal-to-pulse genera un pulso que dispara el temporizador 555 en modo monoestable. A partir de ese momento empieza una ventana temporal durante la cual se espera que el sensor indique la posición alcanzada.

Si el sensor confirma la posición antes de que el temporizador expire, el reset interrumpe el conteo y el sistema considera la operación exitosa. Caso contrario, al agotarse el tiempo establecido, la lógica de alarma interpreta la diferencia entre orden y estado como una falla y activa la señal de error.

Esto nos permite detectar atascamientos mecánicos, fallas eléctricas o
situaciones en las cuales el panel no alcanza la posición esperada dentro
del tiempo previsto.

\section{Sistema de selección y detección de orientación}
\subsection{Latch de selección}
Para seleccionar entre dos estados de funcionamiento (plegado y desplegado) implementamos un latch con transistores NPN acoplados cruzadamente. 
De esta forma, el circuito constitute un bioestable tipo SR, que es capaz de mantener el estado seleccionado hasta recibir un nuevo pulso de conmutación.

La conmutación se realiza mediante pulsos de tensión aplicados a cada una de las entradas. En la simulación, esos pulsos fueron modelados mediante fuentes de voltaje tipo PULSE.

\begin{enumerate}[start=4,label={\alph*.yeah}]
  \item First item
  \item Second item
  \item[custom] Third item
\end{enumerate}

Con la simulación temporal verificamos el correcto funcionamiento del biestable. En ella se observa la conmutación estable entre ambos estados sin oscilaciones indeseadas.

\begin{figure}[H]
  \includegraphics[max width=0.9\linewidth]{graphics/4.schematic.pdf}
  \caption{Esquematico completo}%
  \label{fig:esquematico selector fuente}
\end{figure}


\subsection{Sensor de orientación grosera}
Para determinar la orientación del sistema respecto a la luz solar usamos un detector basado en sensores LDR.

Dispusimos cuatro LDR formando los vértices de un cuadrado, aunque debido a su comportamiento idéntico simulamos solo uno de los canales.

El principio de funcionamiento se basa en utilizar la variación resistiva del LDR en función de la iluminación incidente. En la simulación esta variación fue modelada asignando al LDR una resistencia dependiente de tensión:

\[
R_{LDR} = V(VR) \cdot 10000
\]

donde la fuente auxiliar $V_3$ permite simular un cambio suave en la iluminación.

El circuito usa un comparador para determinar si el nivel de iluminación supera un umbral ajustable mediante un preset modelado por las resistencias $RPOT1$ y $RPOT2$.

\begin{figure}[H]
  \centering
  \includegraphics[width=0.9\linewidth]{graphics/6.schematic.pdf}
  \caption{Circuito del sensor de orientación basado en LDR}
  \label{fig:sensor_orientacion}
\end{figure}

El análisis temporal muestra la variación progresiva del voltaje en el LDR y la correspondiente conmutación en la salida del comparador.

Debido a que la variación lumínica ocurre de manera lenta, no implementamos histéresis en el comparador, ya que no se observaron oscilaciones en la zona de umbral.
\section{Filtro Activo}

\subsection*{Objetivo}
Diseñar, construir y validar experimentalmente un filtro activo pasabajos de segundo orden utilizando un amplificador operacional, con el objetivo de atenuar ruido de alta frecuencia en señales provenientes de sensores satelitales (temperatura, iluminación y corriente).
Se comparó la frecuencia de corte teórica con la medida en el experimento y se analizó el impacto en la calidad de la señal.

\subsection*{Implementacion}
Se utilizó la topología Sallen-Key, ampliamente utilizada para la implementación de filtros activos (op amp) de segundo orden.
El orden dos del sistema se debe a que tenemos dos componentes almacenadores de energía (capacitores $C_1$ y $C_2$), lo que introduce dos polos en la función de transferencia.
La imagen a continuación describe como es la topología en cuestión.

\begin{figure}[H]
  \includegraphics[max width=0.9\linewidth]{graphics/sallenKey.png}
  \caption{Topología Sallen Key}%
  \label{fig:Topología Sallen Key}
\end{figure}

En este caso, para obtener los valores de los componentes decidimos despejarlos a partir de la función transferencia:
\[
H(s)=\frac{K \omega_0^2}{s^2 + \frac{\omega_0}{Q}s + \omega_0^2}
\]

donde:
\[
\omega_0 = \frac{1}{\sqrt{R_1 R_2 C_1 C_2}}
\]

y \(Q\) es el factor de calidad del filtro.

Para simplificar el diseño elegimos una configuración simétrica:

\[
R_1 = R_2
\]
\[
C_1 = C_2
\]

lo que nos permite definir una única frecuencia de corte:

\[
\omega_0 = \frac{1}{RC}
\]

Mediante el seteo de características que necesitábamos, definimos una frecuencia de corte de aproximadamente:
\[
f_c = 100 \ \text{Hz}
\]

lo que equivale a:

\[
\omega_0 = 2\pi f_c \approx 628 \ \text{rad/s}
\]
Con los valores seleccionados:

\[
R = 15.92\,k\Omega, \quad C = 100\,nF
\]

se verifica:

\[
\omega_0 = \frac{1}{RC} \approx 628 \,\text{rad/s}
\]

Buscamos además una ganancia cercana a la unidad en banda pasante.
Por otro lado, la ganancia del amplificador operacional en configuración no inversora está dada por:
\[
K = 1 + \frac{R_4}{R_3}
\]

\subsection*{Simulación}
A continuación, realizamos la simulación de un script en python con la librería \texttt{python-control} para poder visualizar el diagrama de Bode en magnitud y fase a partir de la funcion transferencia del circuito:

\[
H(s)= \frac{1.01}{2.534 \times 10^{-6}s^{2} + 0.001368s + 1}
\]

Estimación de parámetros:
\[
 R_{1}=mR
 \]
 \[
  R_{2}=\frac{R}{m}
 \]
 \[
 C_{1}=nC
  \]
  \[
 C_{2}=C/n
  \]
 \[
 w_{0}=2\cdot\pi\cdot f_{0}
\]
  Investigando al respecto, decidimos que lo más conveniente era definir una única R y una única C para asi poder despejar una única frecuencia de corte. Para poder completar los valores de los esquemáticos escalamos las resistencias y capacitores por las constantes 'm' y 'n' respectivamente.\\
  Probamos valores de resistencias y capacitores que tengan sentido con los materiales a disposición, y, para la magnitud unitaria del Bode, forzamos a que la ganancia en la función transferencia sea 1.01. Dicha ganancia fue obtenida del mismo esquemático en función a las ressitencias:\\
   \[
K=\frac{R_{3}+R_{4}}{R_{3}}=1.01
  \]


\begin{lstlisting}[
    language=python,caption={Simulacion filtro},name=simulacion filtro,label=lst:simulacion filtro
  ]
import control as ct
import sympy as sp
import numpy as np
import matplotlib.pyplot as plt
%matplotlib qt6

R1=15.92*(10**3)
R2=15.92*(10**3)
R3=100*(10**3)
R4=1*(10**3)
C1=100*(10**(-9))
C2=100*(10**(-9))

K=(R3+R4)/R3

num=[0,0,K]
den=[R1*R2*C1*C2,R1*C1+R2*C1+R1*C2*(1-K),1]

H=ct.tf(num, den)

plt.figure()
ct.bode_plot(H,dB=True)

plt.show()
    \end{lstlisting}

\begin{figure}[H]
  \includegraphics[max width=1\linewidth]{graphics/bodeFilter.png}
  \caption{Bode filtro activo}%
  \label{fig:Bode filtro activo}
\end{figure}
De la simulación obtuvimos el Bode en fase y  en magnitud, de los cuales podemos ver que se cumple con el criterio de corte a 200rad\/s  y además se aprecia que a una  superior a 100 rad/s el filtro desfasará nuestra señal de salida respecto a la de entrada.
\subsection*{Pruebas de laboratorio}
Sometimos al filtro a diferentes entradas para poder validar y corroborar su comportamiento descrito por los diagramas de Bode.



\subsubsection*{Modulación AM}
Decidimos probar el filtrado con una senoidal modulada en AM mediante un generador de funciones, dicha portadora contenía altas frecuencias y la envolvente bajas frecuencias para corroborar su debido funcionamiento.
\begin{figure}[H]
  \includegraphics[max width=0.5\linewidth]{graphics/filter1.png}
  \caption{Modulación AM 1}%
  \label{fig:Modulación AM 1}
\end{figure}
El osciloscopio muestra lo que esperábamos, una señal filtrada(amarillo), un poco desfasada en fase y también atenuada en magnitud, la atenuación corresponde a aproximadamente 7 decibeles que es la transición de la frecuencia de corte.
\subsubsection*{Armónicos}
Empleamos el osciloscopio para analizar las componentes frecuenciales de la señal periódica en la señal modulada, a modo de verificar que efectivamente introdujimos unas componentes de alta frecuencia.

\begin{minipage}{0.48\linewidth}
\begin{figure}[H]
  \includegraphics[max width=\linewidth]{graphics/armo.png}
  \caption{Armónicos}%
  \label{fig:Armónicos}
\end{figure}
\end{minipage}
\begin{minipage}{0.48\linewidth}
\begin{figure}[H]
  \includegraphics[max width=\linewidth]{graphics/filter.png}
  \caption{Modulación AM 2}%
  \label{fig:Modulación AM 2}
\end{figure}
\end{minipage}

La segunda figura es una prueba con una entrada modulada a una frecuencia un poco mas baja.

\subsubsection*{Desfasaje}
Ingresamos al sistema con una señal sin modular casi a la frecuencia de corte para visualizar la atenuacion en la magnitud y el cambio de fase correspondiente.
\begin{figure}[H]
  \includegraphics[max width=0.5\linewidth]{graphics/check.png}
  \caption{Desfasaje}%
  \label{fig:desfasaje}
\end{figure}

\section{Control térmico}
Prueba de inserción de control térmico

\begin{enumerate}[start=4,label={\alph*.yeah}]
  \item First item
  \item Second item
  \item[custom] Third item
\end{enumerate}

lalalalalalaaaa
\begin{figure}[H]
  \includegraphics[max width=0.9\linewidth]{graphics/0.board2.png}
  \caption{Example image 1x1}%
  \label{fig:example-image-1x1}
\end{figure}


\bibliographystyle{plain}
\bibliography{refs/example.bib}
\nocite{*}

\end{document}
