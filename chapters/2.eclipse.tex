\section{Orientación solar}
\subsection{Objetivo}

Determinar la orientación del satélite y el evento de un eclipse mediante la luz recibida en sensores luminicos, activando salidas de control correspondientes

\subsection{Implementación}

\subsubsection{Planteo}
Para determinar la orientación del sistema respecto a la luz solar usamos un detector basado en sensores LDR.

Dispusimos cuatro LDR formando los vértices de un cuadrado, aunque debido a su comportamiento idéntico simulamos solo uno de los canales.

El principio de funcionamiento se basa en utilizar la variación resistiva del LDR en función de la iluminación incidente. En la simulación esta variación fue modelada asignando al LDR una resistencia dependiente de tensión:

\[
R_{LDR} = V(VR) \cdot 10000
\]

donde la fuente auxiliar $V_3$ permite simular un cambio suave en la iluminación.

\subsubsection{Consideraciones de funcionamiento}

El circuito usa un comparador para determinar si el nivel de iluminación supera un umbral ajustable mediante un preset modelado por las resistencias $RPOT1$ y $RPOT2$.

El uso de cuatro sensores LDR ubicados en los puntos cardinales permite obtener una estimación de la orientación relativa del satélite respecto a la fuente luminosa.
Una mayor iluminación sobre un sensor implica una menor resistencia del LDR correspondiente, generando una variación en la tensión de entrada del comparador y activando la salida asociada.

La detección de eclipse se define cuando los cuatro sensores presentan niveles
de iluminación por debajo del umbral configurado. En esta condición, las cuatro salidas individuales permanecen en el mismo estado lógico, lo que permite activar una señal global de eclipse.
Debe considerarse que los LDR presentan una respuesta no lineal y dependiente
de la intensidad luminosa, así como variaciones entre dispositivos.
Por este motivo, se incorporan potenciómetros de ajuste que permiten calibrar
los umbrales de disparo y compensar diferencias entre sensores.

De este modo, el sistema permite una detección de condiciones de
iluminación y orientación básica mediante un esquema analógico simple.

\subsubsection{Simulación}
Siendo un circuito muy simple, se realiza unicamente una simulación en la cual se hace variar una resistencia (simulando el LDR) cuyo valor depende de una fuente de voltaje $R = V(VR)\cdot 1M\Omega$. El circuito se muestra a continuación

El análisis temporal muestra la variación progresiva del voltaje en el LDR y la correspondiente conmutación en la salida del comparador.

Debido a que la variación lumínica ocurre de manera lenta, no implementamos histéresis en el comparador, ya que no se observaron oscilaciones en la zona de umbral.

\begin{figure}[H]
  \includegraphics[max width=0.9\linewidth]{graphics/2.1.circuit.png}
  \caption{Simulación del LDR}%
  \label{fig:simulacion LDR}
\end{figure}

La salida es un simple comparador:
\begin{figure}[H]
  \includegraphics[max width=0.9\linewidth]{graphics/2.3.waveform(annotated).png}
  \caption{Salida del LDR}%
  \label{fig:salida LDR}
\end{figure}

\subsubsection{Implementación en el PCB}

Para el esquemático, se usa el mismo circuito comparador 4 veces. Notese que se tienen 4 LEDs, 4 Setpoints y una salida de Eclipse. La detección de eclipse se ajusta mediante un setpoint que se setea al punto en el cual los 4 sensores están apagados (las cuatro salidas en alto)
\begin{figure}[H]
  \includegraphics[max width=0.9\linewidth]{graphics/2.schematic.pdf}
  \caption{Esquematico de detección de orientacion}%
  \label{fig:esquematico orientacion}
\end{figure}


En cuanto al PCB, se agruparon los setpoints para los LDR en un sector junto con el LED indicador de detección de eclipse.
Los 4 sensores se implementan en 4 PCBs separados con conectores para las señales hacia el main board

\begin{minipage}{0.49\textwidth}
\begin{figure}[H]
  \includegraphics[max width=0.9\linewidth]{graphics/2.4.pcb.png}
  \caption{Selectores e indicador de eclipse en el PCB}%
  \label{fig:control eclipse}
\end{figure}
\end{minipage}
\begin{minipage}{0.49\textwidth}
\begin{figure}[H]
  \includegraphics[max width=0.9\linewidth]{graphics/2.5.pcb_sensores.png}
  \caption{Sensores exteriores de luz}%
  \label{fig:sensores exteriores}
\end{figure}
\end{minipage}
