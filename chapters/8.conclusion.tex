\section{Conclusión}
Como conclusión, la implementación de todos los subsistemas nos brindó un panorama integral de lo que implica el diseño electrónico desde sus fundamentos. En este caso trabajamos con electrónica analógica, lo que nos permitió comprender que el flujo de desarrollo parte de una idea inicial, continúa con el diseño del esquemático, la simulación y, como último eslabón, la selección de la tecnología adecuada. A cada etapa le corresponde un equipamiento de laboratorio específico para medir, verificar y validar el funcionamiento.

El diseño de cada esquemático estuvo acompañado por una serie de pruebas y ajustes sucesivos hasta alcanzar el PCB definitivo. En esta instancia final se aplicaron técnicas de optimización y criterios de practicidad que garantizan una correcta interacción entre los distintos bloques del sistema. Sin dudas, fue una experiencia sumamente enriquecedora y satisfactoria.