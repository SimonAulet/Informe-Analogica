\section{Sistema de selección de suministro de corriente}
\subsection{Objetivo}
Diseñar un sistema que mida dos voltajes separados; el de las baterías y el de los paneles solares y conmute entre una fuente y otra de alimentación

\subsection{Implementación}

Se asume un regulador de voltaje y controles varios de potencia aparte.
Este subsistema lo unico que hace es comparar los voltajes y cerrar el circuito que alimenta
la placa de potencia principal usando baterías o paneles solares. Para tal fin se usarán
relés cuyas bobinas se alimentarán con mosfets. El uso de relés se considera óptimo en este
caso por ser extremadamente eficiente con grandes corrientes y tenér relativamente bajas
conmutaciónes.

\subsubsection{Simulaciónes}
\begin{figure}[H]
  \includegraphics[max width=0.99\linewidth]{graphics/5.1.circuit.png}
  \caption{Circuito completo}%
  \label{fig:circuito en LTSpice}
\end{figure}
Para la simulación se reemplaza la bobina del relé con una resistencia.
Voltajes variables simulan las fluctuaciónes (discretas en la simulación)
de las fuentes.

El bloque inversor hace que cuando la salida del comparador está en alto,
un mosfet se activa y cuando está en bajo, lo haga el otro.

\begin{figure}[H]
  \includegraphics[max width=0.99\linewidth]{graphics/5.3.switching(annotated).png}
  \caption{Salida de la simulación}%
  \label{fig:Salida de la simulación}
\end{figure}

En la salida se puede apreciar cómo se ejecuta el switching entre las fuentes. Como
 el comparador tiene histéresis, solo se modifican las salidas cuando la diferencia
 de voltaje es suficiente

\subsubsection{Implementación en PCB}

\paragraph{Esquemático}

 \begin{figure}[H]
   \includegraphics[max width=0.99\linewidth]{graphics/5.schematic.pdf}
   \caption{Esquemático del PCB}%
   \label{fig:Esquemático}
 \end{figure}

El esquemático es igual a la simulación, con la excepción que las resistencias esta vez son los
relés. Se coloca el flyback diode correspondiente.

\paragraph{Estrategia de ruteo}

Al manejar tanta corriente, en lugar de usar pistas se usan áreas enteras del PCB con cobre. Mientras que la señal
de GND está en su plano correspondiente (en In1.Cu) de gran área, las dos señales de batería y solar, se colocan en
B.Cu, resolviendo la ventilación.

En las siguientes imágenes se puede ver tanto el plano del PCB como una captura del renderizado
3D para tener una idea de la dimensión del área de conducción.
\begin{minipage}{0.49\linewidth}
\begin{figure}[H]
  \includegraphics[max width=0.99\linewidth]{graphics/5.4.pcb_zones(annotated).png}
  \caption{Plano anotado de áreas de conducción y GND}%
  \label{fig:Plano pistas}
\end{figure}
\end{minipage}
\begin{minipage}{0.49\linewidth}
\begin{figure}[H]
  \includegraphics[max width=0.99\linewidth]{graphics/5.5.pcb_zones.png}
  \caption{Renderizado 3D de las dos áreas de conducción}%
  \label{fig:Render 3D}
\end{figure}
\end{minipage}

Los relés se colocaron lo más cerca posible de los conectores y lo más alejados posibles del resto de los
circuitos. Es evidente, sin embargo, que no se trata de una buena práctica incluir en este board relés
de tanta corriente. Una mejor práctica sería dejar unicamente los mosfets acá y colocar los relés en el
PCB correspondiente a la gestión de energía o en uno a parte.
