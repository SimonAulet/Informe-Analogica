\section{Sistema de selección y detección de orientación}
\subsection{Latch de selección}
Para seleccionar entre dos estados de funcionamiento (plegado y desplegado) implementamos un latch con transistores NPN acoplados cruzadamente.
De esta forma, el circuito constitute un bioestable tipo SR, que es capaz de mantener el estado seleccionado hasta recibir un nuevo pulso de conmutación.

La conmutación se realiza mediante pulsos de tensión aplicados a cada una de las entradas. En la simulación, esos pulsos fueron modelados mediante fuentes de voltaje tipo PULSE.

\begin{enumerate}[start=4,label={\alph*.yeah}]
  \item First item
  \item Second item
  \item[custom] Third item
\end{enumerate}

Con la simulación temporal verificamos el correcto funcionamiento del biestable. En ella se observa la conmutación estable entre ambos estados sin oscilaciones indeseadas.

\begin{figure}[H]
  \includegraphics[max width=0.9\linewidth]{graphics/4.schematic.pdf}
  \caption{Esquematico completo}%
  \label{fig:esquematico selector fuente}
\end{figure}
