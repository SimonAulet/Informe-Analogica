\section{Control térmico}
\section*{Objetivo}
\section*{Implementación}

\subsection*{Simulaciónes}

\subsubsection*{Planteo matemático}
Para las simulaciónes se parte de la ecuación de schmitt trigger:

\begin{align*}
V_{in}L  &= \frac{R_1}{R_1 + R_2} \left(V_{OL} - V_{ref}\right) + V_{ref} \\
V_{inH}  &= \frac{R_1}{R_1 + R_2}\left(V_{OH} - V_{ref} + V_{ref}\right) \\
H        &= \frac{R_1}{R_1 + R_2} \left(V_{OH} - V_{OL}\right)
\end{align*}

Para elegir los valores, se simula el sistema de ecuaciónes en Python

\begin{lstlisting}[
    language=python,caption={Simulacion histeresis},name=simulacion histeresis,label=lst:simulacion histeresis
  ]

    from sympy import symbols, Eq, solve
    import matplotlib.pyplot as plt

    #Generacion de simbolos
    VinL, VinH, H, R1, R2, VOL, VOH, Vref = symbols('VinL VinH H R1 R2 VOL VOH Vref')

    #Implementacion de ecuaciones
    eq_min = Eq(VinL, R1/(R1+R2) * (VOL - Vref) + Vref)
    eq_max = Eq(VinH, R1/(R1+R2) * (VOH - Vref) + Vref)
    eq_H   = Eq(H,    R1/(R1+R2) * (VOH - VOL))

    #Se sobreescriben los valores conocidos dejandose 3 incognitas
    VinL_val = VinL
    VinH_val = VinH
    H_val    = H
    R1_val   = 5e3
    R2_val   = 100e3
    VOL_val  = 0
    VOH_val  = 9
    Vref_val = 5

    #Se resuelve el circuito para v superior, inferior e histeresis
    sol = solve(
        (eq_min.subs({R1: R1_val, R2:R2_val, VOL:VOL_val, Vref:Vref_val}),
         eq_max.subs({R1: R1_val, R2:R2_val, VOH:VOH_val, Vref:Vref_val}),
         eq_H.subs({  R1: R1_val, R2:R2_val, VOH:VOH_val, VOL:VOL_val})),
        (VinL, VinH, H)
    )
    \end{lstlisting}


\begin{minipage}{0.5\textwidth}
\begin{figure}[H]
  \includegraphics[max width=0.9\linewidth]{graphics/1.6.hysteresis_calculations.png}
  \caption{Ploteo de límites para histéresis}%
  \label{fig:plot-histeresis}
\end{figure}
\end{minipage}
\begin{minipage}{0.4\textwidth}
    El código nos determina límites de voltaje para encendido y apagado ($V_{in}$) de la salida del amplificador, así como los valores superiores e inferior de voltaje que son $VCC$ y $0$ respectivamente. El ancho resulta en $430mV$
\end{minipage}

\subsubsection*{Simulación en LTSPice}
A partir de esto, se realizan las simulacióne en LTSpice. No del circuito final, si no de los tres casos posibles de funcionamiento del schmitt trigger:
\begin{itemize}
    \item Histeresis con inversión
    \item Histéresis sin inversión
    \item Comparador sin histéresis
\end{itemize}

El circuito implementado en LTSpice es el siguiente:

\begin{figure}[H]
  \includegraphics[max width=0.9\linewidth]{graphics/1.1.circuit.png}
  \caption{Circuito implementado en LTSpice}%
  \label{fig:Implementación en LTSpice de Schmitt Trigger}
\end{figure}
Se puede apreciar cómo se hacen 3 circuitos para cada uno de los casos detallados más arriba
\subsubsection*{Resultados simulación}

A continuación, se muestran los resultados de cada simulación realizada. Las primeras tres imágenes muestran la forma de onda de los 3 casos medidos.

\begin{figure}[H]
  \includegraphics[max width=0.9\linewidth]{graphics/1.2.inverting_schmitt.png}
  \caption{Salida de inverting schmitt trigger}%
  \label{fig:Schmitt Trigger inversor}
\end{figure}

\begin{figure}[H]
  \includegraphics[max width=0.9\linewidth]{graphics/1.3.non-inverting_schmitt.png}
  \caption{Salida de non-inverting schmitt trigger}%
  \label{fig:Schmitt Trigger no-inversor}
\end{figure}

\begin{figure}[H]
  \includegraphics[max width=0.9\linewidth]{graphics/1.4.non-hysteretic_comparator.png}
  \caption{Salida sin histéresis}%
  \label{fig:Salida no Schmitt Trigger}
\end{figure}

Finalmente, se incluye una prueba de la amplitud de la histéresis para verificar que la simulación concuerda con los calculos teóricos

\begin{figure}[H]
  \includegraphics[max width=0.9\linewidth]{graphics/1.5.hysteresis_characteristics.png}
  \caption{Medición características de la histéresis}%
  \label{fig:caracteristicas histeresis}
\end{figure}
Se usan marcadores para determinar el ancho de la histéresis $(H)$. Ese valor se resalta en azúl en la ventana de marcadores. Se puede apreciar que da $425mV$, totalmente en línea con lo simulado en Python

\subsection*{Análisis y consideraciones de diseño}
El ancho de histéresis obtenido fue aproximadamente $H \approx 430\,mV$ en la entrada del comparador.
Dado que el sensor LM35 posee una sensibilidad de $10\,mV/^\circ C$ y la etapa amplificadora tiene una ganancia de 5, esto implica que el ancho real de histéresis en temperatura es:

\[
\Delta T = \frac{H}{5 \cdot 10\,mV/^\circ C} \approx 8.6^\circ C
\]

Este valor resulta adecuado para evitar conmutaciones erráticas debidas a pequeñas fluctuaciones o ruido cercano al umbral.

Por otro lado, en la implementación práctica deben considerarse tolerancias de resistencias y no idealidades del LM358, como tensión de saturación no perfectamente rail-to-rail y pequeño offset de entrada.
Estas variaciones pueden generar leves diferencias respecto a los valores teóricos, aunque no afectan demasiado el funcionamiento general del sistema.

En conjunto, el diseño presenta un comportamiento coherente con el análisis teórico, la simulación y la práctica experimental.

\subsection*{Implementación en PCB}
\begin{figure}[H]
  \includegraphics[max width=0.99\linewidth]{graphics/1.schematic.pdf}
  \caption{Esquemático}%
  \label{fig:Esquematico control térmico}
\end{figure}

Para la implementación en el PCB se diseñan 2 bloques diferentes correspondientes a dos laboratorios.
Ambos bloques encienden una señal de control en el board principal y envían una señal de control a la salida.
High temp alert system envía señal de control cuando la temperatura está por encima del valor establecido y low temp heater control hace lo inverso.

\vspace{6pt}
\begin{minipage}{0.3\textwidth}
\flushleft{
El stetpoint de temperatura se establece mediante potenciómetros analógicos ubicados en el main board.
En la figura se muestra el sector del PCB donde se ubican los potenciómetros para setear high y low temp respectivamente más los LED indicadores.}
\end{minipage}%
\begin{minipage}{0.5\textwidth}
\begin{figure}[H]
  \includegraphics[max width=0.99\linewidth]{graphics/1.0.pcb_control.png}
  \caption{Controles manuales en PCB}%
  \label{fig:control manual en PCB}
\end{figure}
\end{minipage}

\subsection*{Evaluación de la etapa de control térmico}

La etapa de control térmico permitió validar el funcionamiento del sistema desde el sensor hasta la señal de salida.
El análisis teórico de los umbrales y del ancho de histéresis mostró coherencia con las simulaciones realizadas, y el armado del modelo confirmó un comportamiento estable del comparador ante variaciones próximas al punto de disparo.

La incorporación de histéresis resultó fundamental para evitar oscilaciones no deseadas, garantizando una conmutación limpia y predecible en condiciones cercanas al umbral.
