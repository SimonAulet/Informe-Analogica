\section{Filtro Activo}

\subsection*{Objetivo}
Diseñar, construir y validar experimentalmente un filtro activo pasabajos de segundo orden utilizando un amplificador operacional, con el objetivo de atenuar ruido de alta frecuencia en señales provenientes de sensores satelitales (temperatura, iluminación y corriente).
Se comparó la frecuencia de corte teórica con la medida en el experimento y se analizó el impacto en la calidad de la señal.

\subsection*{Implementacion}
Se utilizó la topología Sallen-Key, ampliamente utilizada para la implementación de filtros activos (op amp) de segundo orden.
El orden dos del sistema se debe a que tenemos dos componentes almacenadores de energía (capacitores $C_1$ y $C_2$), lo que introduce dos polos en la función de transferencia.
La imagen a continuación describe como es la topología en cuestión.

\begin{figure}[H]
  \includegraphics[max width=0.9\linewidth]{graphics/sallenKey.png}
  \caption{Topología Sallen Key}%
  \label{fig:Topología Sallen Key}
\end{figure}

En este caso, para obtener los valores de los componentes decidimos despejarlos a partir de la función transferencia:
\[
H(s)=\frac{K \omega_0^2}{s^2 + \frac{\omega_0}{Q}s + \omega_0^2}
\]

donde:
\[
\omega_0 = \frac{1}{\sqrt{R_1 R_2 C_1 C_2}}
\]

y \(Q\) es el factor de calidad del filtro.

Para simplificar el diseño elegimos una configuración simétrica:

\[
R_1 = R_2
\]
\[
C_1 = C_2
\]

lo que nos permite definir una única frecuencia de corte:

\[
\omega_0 = \frac{1}{RC}
\]

Mediante el seteo de características que necesitábamos, definimos una frecuencia de corte de aproximadamente:
\[
f_c = 100 \ \text{Hz}
\]

lo que equivale a:

\[
\omega_0 = 2\pi f_c \approx 628 \ \text{rad/s}
\]
Con los valores seleccionados:

\[
R = 15.92\,k\Omega, \quad C = 100\,nF
\]

se verifica:

\[
\omega_0 = \frac{1}{RC} \approx 628 \,\text{rad/s}
\]

Buscamos además una ganancia cercana a la unidad en banda pasante.
Por otro lado, la ganancia del amplificador operacional en configuración no inversora está dada por:
\[
K = 1 + \frac{R_4}{R_3}
\]

\subsection*{Simulación}
A continuación, realizamos la simulación de un script en python con la librería \texttt{python-control} para poder visualizar el diagrama de Bode en magnitud y fase a partir de la funcion transferencia del circuito:

\[
H(s)= \frac{1.01}{2.534 \times 10^{-6}s^{2} + 0.001368s + 1}
\]

Estimación de parámetros:
\[
 R_{1}=mR
 \]
 \[
  R_{2}=\frac{R}{m}
 \]
 \[
 C_{1}=nC
  \]
  \[
 C_{2}=C/n
  \]
 \[
 w_{0}=2\cdot\pi\cdot f_{0}
\]
  Investigando al respecto, decidimos que lo más conveniente era definir una única R y una única C para asi poder despejar una única frecuencia de corte. Para poder completar los valores de los esquemáticos escalamos las resistencias y capacitores por las constantes 'm' y 'n' respectivamente.\\
  Probamos valores de resistencias y capacitores que tengan sentido con los materiales a disposición, y, para la magnitud unitaria del Bode, forzamos a que la ganancia en la función transferencia sea 1.01. Dicha ganancia fue obtenida del mismo esquemático en función a las ressitencias:\\
   \[
K=\frac{R_{3}+R_{4}}{R_{3}}=1.01
  \]


\begin{lstlisting}[
    language=python,caption={Simulacion filtro},name=simulacion filtro,label=lst:simulacion filtro
  ]
import control as ct
import sympy as sp
import numpy as np
import matplotlib.pyplot as plt
%matplotlib qt6

R1=15.92*(10**3)
R2=15.92*(10**3)
R3=100*(10**3)
R4=1*(10**3)
C1=100*(10**(-9))
C2=100*(10**(-9))

K=(R3+R4)/R3

num=[0,0,K]
den=[R1*R2*C1*C2,R1*C1+R2*C1+R1*C2*(1-K),1]

H=ct.tf(num, den)

plt.figure()
ct.bode_plot(H,dB=True)

plt.show()
    \end{lstlisting}

\begin{figure}[H]
  \includegraphics[max width=1\linewidth]{graphics/bodeFilter.png}
  \caption{Bode filtro activo}%
  \label{fig:Bode filtro activo}
\end{figure}
De la simulación obtuvimos el Bode en fase y  en magnitud, de los cuales podemos ver que se cumple con el criterio de corte a 200rad\/s  y además se aprecia que a una  superior a 100 rad/s el filtro desfasará nuestra señal de salida respecto a la de entrada.
\subsection*{Pruebas de laboratorio}
Sometimos al filtro a diferentes entradas para poder validar y corroborar su comportamiento descrito por los diagramas de Bode.



\subsubsection*{Modulación AM}
Decidimos probar el filtrado con una senoidal modulada en AM mediante un generador de funciones, dicha portadora contenía altas frecuencias y la envolvente bajas frecuencias para corroborar su debido funcionamiento.
\begin{figure}[H]
  \includegraphics[max width=0.5\linewidth]{graphics/filter1.png}
  \caption{Modulación AM 1}%
  \label{fig:Modulación AM 1}
\end{figure}
El osciloscopio muestra lo que esperábamos, una señal filtrada(amarillo), un poco desfasada en fase y también atenuada en magnitud, la atenuación corresponde a aproximadamente 7 decibeles que es la transición de la frecuencia de corte.
\subsubsection*{Armónicos}
Empleamos el osciloscopio para analizar las componentes frecuenciales de la señal periódica en la señal modulada, a modo de verificar que efectivamente introdujimos unas componentes de alta frecuencia.

\begin{minipage}{0.48\linewidth}
\begin{figure}[H]
  \includegraphics[max width=\linewidth]{graphics/armo.png}
  \caption{Armónicos}%
  \label{fig:Armónicos}
\end{figure}
\end{minipage}
\begin{minipage}{0.48\linewidth}
\begin{figure}[H]
  \includegraphics[max width=\linewidth]{graphics/filter.png}
  \caption{Modulación AM 2}%
  \label{fig:Modulación AM 2}
\end{figure}
\end{minipage}

La segunda figura es una prueba con una entrada modulada a una frecuencia un poco mas baja.

\subsubsection*{Desfasaje}
Ingresamos al sistema con una señal sin modular casi a la frecuencia de corte para visualizar la atenuacion en la magnitud y el cambio de fase correspondiente.
\begin{figure}[H]
  \includegraphics[max width=0.5\linewidth]{graphics/check.png}
  \caption{Desfasaje}%
  \label{fig:desfasaje}
\end{figure}
